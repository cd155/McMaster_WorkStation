\documentclass[12pt]{article}

\usepackage{graphicx}
\usepackage{paralist}
\usepackage{listings}
\usepackage{booktabs}
\usepackage{hyperref}

\oddsidemargin 0mm
\evensidemargin 0mm
\textwidth 160mm
\textheight 200mm

\pagestyle {plain}
\pagenumbering{arabic}

\newcounter{stepnum}

\title{Greedy Algorithm Drasil Case Study}
\author{Don Chen}
\date{\today}

\begin {document}

\maketitle
This is the proposal of greedy algorithms/heuristics to my drasil case study.

\section{Problem Description}
\begin{itemize}
    \item Use greedy heuristics to find the "smallerst" number in a tree. 
    The algorithm can be summarized as:
    \begin{itemize}
        \item At each node, pick out the note contains the smallest result.
        \item Repeat first step until reachs to any leaf.
    \end{itemize}
    \item The Traveling Salesman Problem(TSP): 
    Find the shortest distance tour passing through each node of the network exactly once. 
    \begin{itemize}
        \item Nearest Neighbor Alorithm
        \begin{itemize}
            \item Start at any city
            \item Visit the nearest node not yet visited
            \item Return to the start node when all other nodes are visited
        \end{itemize}
        \item Kruskal’s Algorithm
        \begin{itemize}
            \item Select the miminum weight of edge if the edge doesn't form a circle
            \item Stop when all node is connected
            \item \href{https://en.wikipedia.org/wiki/Kruskal%27s_algorithm}{Kruskal’s Algorithm Wiki}
        \end{itemize}
    \end{itemize}
\end{itemize}

\section{Goal Statements}

\begin{itemize}
    \item Find relatively a good solution with less costly computation. 
    This maybe too general, how to define a good solution?
    \item Find the optimal solutiuon for The Traveling Salesman Problem.
\end{itemize}

\section{Assumptions}
Assume all type are number.
Greedy heuristics work well in a quantified environment, but not all type can be quantified.
Some type could be hard to be quantified, like boolean. 
In decision tree, some nodes will ask yes or no quesiton.

\section{Theoretical Model}
\section{General Definition}
G is a connected graph with n vertices, T is minimal connected spanning of G.
\section{Supporting Data Definitions}

\section{Instance Models}
\begin{itemize}
    \item Tree
    \begin{itemize}
        \item Input: Start at the top node
        \item Output: Number
    \end{itemize}
    \item Traveling Salesman Problem
    \begin{itemize}
        \item Input: Start anywhere
        \item Output: Number
    \end{itemize}
\end{itemize}

\section{Question}
\begin{itemize}
    \item (may not relevant now)Greedy algorithms/heuristics is more like a technique rather than an algorithm. Are we looking for two 
    specific algorithm? The reason I aked this question is that it is eaiser to list a problem, then try to find a algorithm 
    to solve the problem.
    \item greedy heuristics impact on wild animal. Animal cognitive, Migrate bird. 
    \href{https://www.researchgate.net/publication/263430731_Practice_makes_proficient_pigeons_Columba_livia_learn_efficient_routes_on_full-circuit_navigational_traveling_salesperson_problems}{Pigeon experiment}
\end{itemize}

\section{Reference}
\begin{itemize}
    \item \href{http://math.mit.edu/~goemans/18433S15/TSP-CookCPS.pdf}{Nearest Neighbor Alorithm} 
    \item \href{https://www.ams.org/journals/proc/1956-007-01/S0002-9939-1956-0078686-7/home.html}{Kruskal  Alorithm in TSP} 
\end{itemize}

\end {document}