\documentclass[12pt]{article}

\usepackage{graphicx}
\usepackage{paralist}
\usepackage{listings}
\usepackage{booktabs}
\usepackage{hyperref}

\oddsidemargin 0mm
\evensidemargin 0mm
\textwidth 160mm
\textheight 200mm

\pagestyle {plain}
\pagenumbering{arabic}

\newcounter{stepnum}

\title{Greedy Algorithm Drasil Case Study}
\author{Don Chen}
\date{\today}

\begin {document}

\maketitle
This is the proposal of the family of greedy algorithms to my drasil case study.

\section{Problem Description}
\begin{itemize}
    \item Given a list of cities and the distances between each pair of cities, 
    what is the shortest possible route that visits each city exactly once?
\end{itemize}

\section{Goal Statements}
\begin{itemize}
    \item General: Find the shortest possible route that visits each city exactly once and returns to the origin city
    \item Mathematicl Definition: In the weighted graph, find the shorest path which contains all vertices.
\end{itemize}

\section{Assumptions}
\begin{itemize}
    \item V and E are finite sets. 
    \item It is undirected graphs, means f(e) = uv = vu.
    \item It is a connected graph.
    \item There is no loop in the graph.
\end{itemize}

\section{Theoretical Model}
\begin{itemize}
    \item Nearest Neighbor Alorithm
    \item Kruskal’s Algorithm
\end{itemize}

\section{Defined In Graphic}
\begin{itemize}
    \item Each city is a node, and each node will has its name.
    \item A node can connect to other nodes with a line.
    \item Once two nodes connected, there will be a distance assgined on the line.
\end{itemize}

\section{General Definition}
A graph G is an ordered triple (V(G), E(G), f) consisting of a nonempty set V(G) of vertices, a set E(G) of edges, 
and an incidence function f that associates with each edge and two vertices. If e is an edge and u and v are vertices, 
u, v $\in$  V (G), such that f(e) = uv, then e is said to join u and v; the vertices u and v are called the ends of e. u 
and v are endpoints of edge e. A weighted graph is a graph with each edge e of G let there be associated a real number w(e).
\begin{itemize}
    \item G = (V(G), E(G), f, w)
    \item V(G) = {$v\_1$, v\_2, v\_3, v\_4, v\_5}, E(G) = {e\_1, e\_2, e\_3, e\_4, e\_5, e\_6, e\_7, e\_8}
    \item f(e\_1) = v\_1v\_2, f(e\_2) = v\_2v\_3, f(e3) = v\_3v\_3, f(e4) = v\_3v\_4, f(e5) = v\_2v\_4
    f(e\_6) = v\_4v\_5, f(e\_7) = v\_2v\_5, f(e\_8) = v\_2v\_5
    \item w(e\_1, e\_2, e\_3, e\_4, e\_5, e\_6, e\_7, e\_8) = n\_1, n\_2, n\_3, n\_4, n\_5, n\_6, n\_7, n\_8
\end{itemize}

\section{Supporting Data Definitions}
\begin{itemize}
    \item Finite Graph: A graph is finite if both its vertex set and edge set are finite.
    \item Endpoint: if f(e) = uv, edge e has endpints of u and v. u and v $\in$  V(G).
    \item Loop: An edge with identical ends. f(e) = vu and v = u. u and v $\in$  V(G).
    \item Connected Graph: if for every u, v $\in$  V(G) there exists f(e) = uv. Otherwise G is called disconnected.
    \item Walk: A walk is a sequence W = {v\_0, e\_1, v\_1, $...$, e\_k, v\_k}, whose terms are alternatively vertices 
    and edges such that for 1$\le$i$\le$k, the edge e\_i has endpoints v\_{i$-$1} and v\_i. k is the length of W.
    \item Trail: If the edges e1, e2, $...$, ek of a walk are distinct.
    \item Path: a path is a walk if both of its edges e1, e2, $...$, ek and its vertices v\_0, v\_1, $...$ , v\_k are distinct.
    \item Closed: A u,v-walk has first vertex u and last vertex v. 
    When the first and last vertex of a walk, trail or path are the same, we say that they are closed. 
    \item The Shorest Path: a path has the smallest number of sum of its edges' weight.
    \item Weight: With each edge e of G let there be associated a real number w(e), called its weight.
\end{itemize}

\section{Instance Models}
\begin{itemize}
    \item Input: G = (V(G), E(G), f, w)
    \item Output: a sequence
\end{itemize}

\section{Question}
\begin{itemize}
    \item remove the conditon of "returns to the origin city"
\end{itemize}

\section{Reference}
\begin{itemize}
    \item Graph Theory With Applications by J.A. Bondy and U. S. R. Murty
    \item \href{http://math.mit.edu/~goemans/18433S15/TSP-CookCPS.pdf}{Nearest Neighbor Alorithm} 
    \item \href{https://www.ams.org/journals/proc/1956-007-01/S0002-9939-1956-0078686-7/home.html}{Kruskal  Alorithm in TSP} 
\end{itemize}

\end {document}