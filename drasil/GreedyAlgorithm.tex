\documentclass[12pt]{article}

\usepackage{graphicx}
\usepackage{paralist}
\usepackage{listings}
\usepackage{booktabs}
\usepackage{hyperref}

\oddsidemargin 0mm
\evensidemargin 0mm
\textwidth 160mm
\textheight 200mm

\pagestyle {plain}
\pagenumbering{arabic}

\newcounter{stepnum}

\title{Greedy Algorithm Drasil Case Study}
\author{Don Chen}
\date{\today}

\begin {document}

\maketitle
This is the proposal of greedy algorithms/heuristics to my drasil case study.

\section{Problem Description}
\begin{itemize}
    \item Use greedy heuristics to find the "smallerst" number in a tree. 
    The algorithm can be summarized as:
    \begin{itemize}
        \item At each node, pick out the note contains the smallest result.
        \item Repeat first step until reachs to any leaf.
    \end{itemize}
    \item Use greedy heuristics in the Traveling Salesman Problem(TSP): 
    Find the shortest distance tour passing through each node of the network exactly once. Nearest Neighbor Alorithm:
    \begin{itemize}
        \item Start at any city
        \item Visit the nearest node not yet yet visited
        \item Return to the start node when all other nodes are visited
    \end{itemize}
\end{itemize}

\section{Goal Statements}
Find relatively good solution with less costly way.
\begin{itemize}
    \item This maybe too general, how to define a good solution
\end{itemize}

\section{Assumptions}
All types in tree or graph could be quantified. Some type could be hard to be quantified, like boolean. In some decision tree,
some nodes will ask yes or no quesiton.

\section{Theoretical Model}
\section{General Definition}
\section{Supporting Data Definitions}

\section{Instance Models}
\begin{itemize}
    \item Tree
    \begin{itemize}
        \item Input: Start node
        \item Output: Number
    \end{itemize}
    \item TSP
    \begin{itemize}
        \item Input: Start city
        \item Output: Number
    \end{itemize}
\end{itemize}

\section{Question}
\begin{itemize}
    \item Greedy algorithms/heuristics is more like a technique rather than an algorithm. Are we looking for two 
    specific algorithm? The reason I aked this question is it is eaiser to list a problem, then try to find a algorithm 
    to solve the problem. 
\end{itemize}

\section{Reference}
\begin{itemize}
    \item \href{http://math.mit.edu/~goemans/18433S15/TSP-CookCPS.pdf}{Nearest Neighbor Alorithm} 
\end{itemize}

\end {document}