\documentclass[12pt]{article}

\usepackage{graphicx}
\usepackage{paralist}
\usepackage{listings}
\usepackage{booktabs}
\usepackage{hyperref}

\oddsidemargin 0mm
\evensidemargin 0mm
\textwidth 160mm
\textheight 200mm

\pagestyle {plain}
\pagenumbering{arabic}

\newcounter{stepnum}

\title{Greedy Algorithm Drasil Case Study}
\author{Don Chen}
\date{\today}

\begin {document}

\maketitle
This is the proposal of greedy algorithms/heuristics to my drasil case study.

\section{Problem Description}
\begin{itemize}
    \item Use greedy heuristics to find the "smallerst" number in a tree. 
    The algorithm can be summarized as:
    \begin{itemize}
        \item At each node, pick out the note contains the smallest result.
        \item Repeat first step until reachs to any leaf.
    \end{itemize}
    \item The Traveling Salesman Problem(TSP): 
    Find the shortest distance tour passing through each node of the network exactly once. 
    \begin{itemize}
        \item Nearest Neighbor Alorithm
        \item Kruskal’s Algorithm
    \end{itemize}
\end{itemize}

\section{Goal Statements}

\begin{itemize}
    \item Find relatively a good solution with less costly computation. 
    This maybe too general, how to define a good solution?
    \item Find the optimal solutiuon for The Traveling Salesman Problem.
\end{itemize}

\section{Assumptions}

\begin{itemize}
    \item V and E are finite sets. 
    \item It is undirected graphs, means (u,v) = (v,u).
\end{itemize}

\section{Theoretical Model}
\section{General Definition}
A graph G is an ordered triple (V(G), E(G), f) consisting of a nonempty set V(G) of vertices, a set E(G) of edges, 
and an incidence function f that associates with each edge and two vertices. If e is an edge and u and v are vertices
such that f(e) = uv, then e is said to join u and v; the vertices u and v are called the ends of e. u and v are endpoints 
of edge e.
\begin{itemize}
    \item Adjacency: Vertices u, v ∈ V are said adjacent if joined by an edge in E
    \item Degree: The degree dv of vertex v is its number of incident edges
    \item It is undirected graphs, means (u,v) = (v,u).
\end{itemize}
\section{Supporting Data Definitions}

\section{Instance Models}
\begin{itemize}
    \item Tree
    \begin{itemize}
        \item Input: Start at the top node
        \item Output: Number
    \end{itemize}
    \item Traveling Salesman Problem
    \begin{itemize}
        \item Input: Start anywhere
        \item Output: Number
    \end{itemize}
\end{itemize}

\section{Question}
\begin{itemize}
    \item (may not relevant now)Greedy algorithms/heuristics is more like a technique rather than an algorithm. Are we looking for two 
    specific algorithm? The reason I aked this question is that it is eaiser to list a problem, then try to find a algorithm 
    to solve the problem.
    \item greedy heuristics impact on wild animal. Animal cognitive, Migrate bird. 
    \href{https://www.researchgate.net/publication/263430731_Practice_makes_proficient_pigeons_Columba_livia_learn_efficient_routes_on_full-circuit_navigational_traveling_salesperson_problems}{Pigeon experiment}
\end{itemize}

\section{Reference}
\begin{itemize}
    \item \href{http://math.mit.edu/~goemans/18433S15/TSP-CookCPS.pdf}{Nearest Neighbor Alorithm} 
    \item \href{https://www.ams.org/journals/proc/1956-007-01/S0002-9939-1956-0078686-7/home.html}{Kruskal  Alorithm in TSP} 
\end{itemize}

\end {document}